\documentclass[11pt]{report}
\usepackage[utf8]{inputenc}
\usepackage{mathptm}
\usepackage{newcent}
\usepackage[letterpaper]{geometry}
\usepackage{graphicx}
\usepackage[perpage]{footmisc}
\usepackage{wasysym}
\usepackage{amssymb}
\usepackage{footnote}
\usepackage{enumitem}
\def\topfraction{.99}
\def\textfraction{.01}
\def\floatpagefraction{1}
\makesavenoteenv{tabular}

\def\Ohm{$\Omega$}
\def\subfig#1#2{{\sf\small\bfseries#1}\raisebox{-.2in}{\raisebox{-.5\height}{\includegraphics[scale=.6]{#2}}}}
\def\subfigh#1#2#3{{\sf\small\bfseries#1}\raisebox{.1in}{\raisebox{-\height}{\includegraphics[height=#3]{#2}}}}
\def\sub#1{\textrm{\scriptsize#1}}

\begin{document}
\thispagestyle{empty}
\begin{centering}
  {\Huge CPCB}
  \vskip30pt

  {\large PCB design for scientists and other part-time electrical engineers}
  \vskip60pt

  {\large Daniel A. Wagenaar}
  \vfill
  
  {Copyright (c) 2018}
  
\end{centering}
\pagebreak
~
\vfill
\noindent Copyright (C) 2018 Daniel A. Wagenaar\medskip

``CPCB'' is free software: you can redistribute it and/or modify
it under the terms of the GNU General Public License as published by
the Free Software Foundation, either version 3 of the License, or
(at your option) any later version.

This program is distributed in the hope that it will be useful,
but WITHOUT ANY WARRANTY; without even the implied warranty of
MERCHANTABILITY or FITNESS FOR A PARTICULAR PURPOSE.  See the
GNU General Public License for more details.

You should have received a copy of the GNU General Public License
along with this program.  If not, see http://www.gnu.org/licenses.
\pagebreak

\chapter{Introduction}

This document describes the installation and usage of ``CPCB,'' a
(hopefully) easy-to-use program for designing printed circuit board
(PCBs) written by Daniel Wagenaar. A companion program, ``CSchem,'' is
provided for designing circuits prior to instantiating them on a
PCB. This program is documented separately.

\section{Why use CPCB?}

There are any number of software packages and online options available
that allow you to design PCBs. So why should you choose CPCB? CPCB is
for you if:
\begin{itemize}
\item You like a program that keeps simple tasks simple;
\item You just want to make a few boards for a project and get on
  with your life;
\item You like a what-you-see-is-what-you-get approach;
\item You like to switch between imperial and metric units fluidly;
\item You use CSchem for drawing circuits.        
\end{itemize}
%
On the other hand, CPCB might not be for you if:
\begin{itemize}
  \item You need to specify arcane properties on
  your electronic parts for automated workflows;
\item Your projects are so complex that you really need an
  autorouter;
\item You need boards with internal layers.
\end{itemize}
%
Lastly, a note on development. CPCB is being developed by an
active research scientist. Practically, that means two things: On the
positive side, it means that I have a vested interest in fixing bugs
and improving CPCB, because I use it regularly. On the negative side, that
means that, by and large, new features are added only when I need them
and bugs are fixed when I have time. I certainly do welcome feature
requests, but I cannot guarantee that they will get implemented
quickly or at all. (If you are in a hurry, I will consider (paid)
consultancy related to CPCB.) Finally, I definitely welcome
contributions to either the code or the documentation. I would be very
happy if CPCB turned into a community-supported open source project.

\section{Features}

A document in CPCB consists of a single PCB that can be either
rectangular or round.\footnote{More complex outline shapes are planned
  for the future.} The PCB can be populated by traces, pads, holes,
and text.\footnote{There are two more basic objects, arcs and filled
  planes, which will be discussed in  Chapter~\ref{ch.using}.} These basic
objects can be placed on any of three layers: a bottom copper layer, a
top copper layer, and a top silkscreen layer. Most of the time, holes
and pads on the copper layers will be grouped together with some text
and line segments on the silkscreen layer to form a ``component,''
such as a connector, a resistor, or an IC socket.

Objects have various properties, such as line width, hole diameter, or
pad size, and these can be edited by means of a toolbar on the right
of the main window. Objects can be placed on the PCB, connected with
traces, moved around, rotated, transferred to other layers, etc., all
with straightforward mouse interactions.

CPCB comes with a small library of predefined components, but creating
your own custom components is very easy thanks to a flexible system of
coordinates and grids and quick access to functions to label pins.

CPCB can be used stand-alone for small projects, but is a particularly
strong tool in combination with CSchem: If you ``link'' a CSchem
schematic to a PCB layout, CPCB will show a toolbar with all the
components that you have not yet placed, and will also highlight
connections that are required to complete the circuit as designed.

CPCB can export its designs in Gerber format, which is the
\emph{lingua franca} that most online PCB fabrication services
understand.

\section{Contacting the author}

If you like CPCB or find fault with it, if you discover a bug or have a
suggestion for a new feature, if you are interested in improving this
documentation or have a patch to contribute to the code, I want to
hear from you. My contact information is at
http://www.danielwagenaar.net. I very much look forward to hearing
from you. I realize that this guide is extremely terse, and I
really do welcome questions, particularly if they help me to improve
CPCB or its documentation.\bigskip

\noindent Pasadena, October 2018

\chapter{Installation}\label{ch.install}

The latest version of the software can always be downloaded from\break
http://www.danielwagenaar.net/cschem.

\section{Precompiled binaries}

Binary packages for Linux, Windows and Mac OS X will be provided as time
permits. You can help focus my attention on binaries simply by
expressing an interest.

% Installation on Windows should be easy using the provided ``CSchem.msi''
% installation package. Installation on Mac OS X should be
% straightforward by unpacking the ``CSchem-mac.tgz'' archive and placing
% ``CSchem.app'' anywhere on your hard disk.  Installation on Debian,
% Ubuntu, or Mint Linux should be equally easy using the provided
% ``CSchem.deb'' installation package. At present, installation on other
% flavors of Linux will require compiling the sources yourself, but this
% should be straightforward (see below).

% Please note that development occurs
% primarily on Linux, so the Windows and Mac OS versions may lag
% behind.

\section{Compiling the source}
To compile the source,  start from the provided
``cschem.tar.gz'' archive or check out the git source at
http://github.com/wagenadl/cschem. Compilation requires
``Qt'' version 5.6 or later.

\subsection{Compiling on Linux or Mac OS}

You will need a C++ compiler and ``make''. On Ubuntu Linux, this is as simple
as ``sudo apt-get install g++ make''. On Mac OS, you need the
``Command Line tools for XCode'' from the Apple Developers' web
site\footnote{https://developer.apple.com/xcode.}.

Open a terminal and ``cd'' to the root of the unpacked source
archive. Then type ``make'' and fetch a cup of tea. Then, either
manually copy the file ``build/cpcb/cpcb to some location on your PATH, or type ``sudo make
install'' to install into ``/usr/local/bin''.

\subsection{Compiling on Windows}

Details to follow. Again, feel free to ask!
% You will need a C++ compiler. I have successfully used both MinGW and
% Microsoft Visual Studio.
% 
% First, run ``updatesources.sh'' in the
% ``tools'' subfolder in a Cygwin shell. Then open, one by one,
% ``src/CSchem.pro''
% and ``webgrab/webgrab.pro'' in Qt
% Creator and follow the standard build steps.

\chapter{Using CPCB}\label{ch.using}

\section{The graphical user interface}

CPCB has a relatively spare user interface with several toolbars
around the PCB being edited. These toolbars are:
\begin{description}
  \item[Mode bar] CPCB makes heavy use of
``editing modes.'' At any time, CPCB is in one of several modes, and a
toolbar on the left of the main window is used to switch between modes
and to indicate the current mode.
\item[Properties bar]  A toolbar on the right shows the various
  editable properties of
  selected objects or objects to be placed.
\item[Status bar] A small  bar at the bottom of the window shows cursor
  position, grid settings, and which features are currently being shown.
\item[Menu bar]Along the top of the window a standard menu bar is
  shown.
\end{description}
One other toolbar is shown only  when CPCB is used to edit a PCB design that
is linked to a CSchem schematic:
\begin{description}
  \item[Components pane] Here are shown all the not-yet-placed
    components. You can drag component footprints from a Filer
    window\footnote{\footnote{E.g., Gnome Files in Linux, the Finder
        in Mac OS, or the File Explorer in Windows.} into this pane,
      or from this pane onto the main PCB.
\end{description}

\subsection{Editing modes}
Because changing modes is such a frequent task,
keyboard shortcuts are available for each of the modes. The modes are
(from top to bottom):
\begin{description}
\item[Selection mode (F1)]This is a highly multifunctional mode:
  \begin{itemize}
  \item Objects can be selected by
    clicking on them or dragging a rectangle over them.
  \item Selected objects can be modified using the Properties bar.
  \item Objects can be moved by dragging with the mouse.
  \item Objects can be deleted or placed on a clipboard using
    standard cut-and-paste operations.
  \item Selected objects can be grouped into a component
    (``Control''+``G'') and existing components can be ungrouped
    into separate objects (``Control''+``Shift''+``G'').
  \end{itemize}
  In addition, double-clicking on different types of objects has
  specific effects in this mode:
  \begin{itemize}
    \item Double-clicking on a group ``enters'' the group so its
      contents can be edited in-place. (Double-clicking on blank space
      ``exits'' a group that was entered in this way.)
    \item Double-clicking on a pad or hole allows the name or number
      of that pad or hole to be changed.\footnote{When a pad or hole
        is part of a component that is linked to an element on a
        CSchem schematic, a combobox with predefined choices appears;
        otherwise, a free-form name editor appears.}
      \item Double-clicking on a piece of text allows the text to be
        edited.
  \end{itemize}
  \item[Trace mode (F2)] In this mode, new traces are placed on the
    PCB by clicking with the mouse. The Properties bar can be used to determine the width of the
    traces as well as which layer they will be placed on. Traces can
    be placed on copper and silkscreen layers alike.
    \item[Plated hole mode (F3)] In this mode, mouse clicks place
      plated holes on the board. Holes always connect the top and
      bottom layers of the PCB. The Properties bar can be used to
      determine the hole diameter and the diameter of the surrounding
      pad. Holes can have either round or rectangular pads. Instead of
      a simple drill hole, straight milled slots can be placed in this
      mode as well; the slot length is determined on the Properties
      bar. Usually, holes are part of components, in which case the
      ``Pin'' box in the Properties bar shows the pin name or number
      for the hole. This mode should \emph{not} be used to create
      vias; see below.
\item[Pad mode (F4)] In this mode, copper pads are placed on the
  PCB. Pads are generally used as part of surface-mount components, in
  which case the ``Pin'' box in the Properties bar shows the pin name
  or number for the pad.  Pads are normally placed only on copper
  layers, but CPCB doesn't stop you if you want to place a pad on the
  silkscreen layer.
\item[Text mode (F5)] In this mode, clicking on the board opens a
  small dialog window in which you type the text to be placed. Text is
  normally placed on the silkscreen layer, but may be placed on copper
  layers as well.
  \item[Arc mode (F6)] In this mode, circles or arcs are placed,
    typically on the silkscreen mode as part of component outline
    drawings.
    \item[Filled plane mode (F7)] Filled planes are filled areas on
      either copper layer. In filled plane mode, new planes can be
      created, existing planes can be edited, and holes and pads can
      be connected or disconnected from filled planes by double
      clicking. Use the Layer selector in the Properties bar
      (\emph{not} the Layer visibility toggles in the status bar) to
      determine the layer for a filled plane or filled plane
      connection.
      \item[Pickup mode (F8)] In this mode, clicking on an existing
        trace picks up that trace so that you can conveniently
        reconnect it to another point on the PCB.
      \item[Nonplated hole mode (F9)] Nonplated holes are holes
        without copper plating. This mode can also be used to create
        straight milled slots.
      \item[Board outline mode (F10)] In future versions, this
        mode will allow editing of the board outline in arbitrary
        shapes, but at present this is not yet implemented.
\end{description}
The Mode bar contains two additional buttons that do not correspond to
editing modes:
\begin{description}
      \item[Origin selection (F11)] When CPCB first starts, it uses
        absolute coordinates with (0,0) at the top-left of the
        board. Press F11 switches to incremental coordinates and
        invites the user to click on a hole or pad to set the
        origin for incremental coordinates. After an incremental
        origin is selected, CPCB automatically reverts to
        Selection mode. To return to absolute coordinates, press
        F11 again. (When entering incremental coordinate mode,
        you don't \emph{have} to pick a new origin; just press
        F1 to return to Selection mode with the previous
        incremental origin reinstated.)
      \item[Angle constraint (F12)] Use this to toggle between placing
        traces with arbitrary angles and placing traces that are
        either horizontal, vertical, or at 45$^\circ$ to the canonical
        axes.
\end{description}

\subsection{The Properties bar}
Although most items in the Properties bar may speak for themselves,
here is a list with a few notes. From top to bottom:
\begin{description}
\item[X] Shows the x-coordinate of the center of a hole, pad, or
  arc, or the left edge of a piece of text. When multiple objects
  are selected, the leftmost x-coordinate is shown.\footnote{At present, this
    is true even when those objects are contained in a group. In a
    future version, the x-coordinate of a group may be defined as the
    x-coordinate of its lowest numbered pin.} You can type in the box
  to move the selection to a new location.
\item[Y] The y-coordinate equivalent of previous item. When
  multiple objects are selected, the topmost
  y-coordinate is shown.
\item[Line width] This item is meaningful in relation to
  traces and arcs.
\item[Diameter] The diameter of (plated and nonplated) holes as well
  as arcs.
\item[Slot length] The extra length of a milled slot. (The full
  length of a slot is the stated ``Slot length'' plus the diameter.)
  Meaningful for plated and nonplated holes.
\item[OD] Outer diameter of the round pad surrounding a plated hole, or
  length of the square pad surrounding such a hole. If the hole is in
  fact a slot, the pad is stretched along with the hole.
\item[Shape] Selects whether the pad surrounding a hole is round or
  rectangular.\footnote{``Rectangular'' really means ``square''
    except if the hole has nonzero slot length.}
\item[W] The width of an SMT pad (along the x-axis).
\item[H] The height of an SMT pad (along the y-axis).
\item[Ref. {\normalfont\itshape or}~Pin {\normalfont\itshape
    or}~Text] Multifunctional item that shows the ``Reference'' ID for
  a component; the ``Pin number'' for a hole or pad; or the wording of
  a text object.
\item[Fs] Font size for textual annotations
\item[Arc style] Selector for different kinds of arcs
\end{description}
%
Wherever dimensions appear in the Properties bar, simple arithmetic is
accepted. Thus you can type things like ``0.3 in + 2*0.7 cm''. CPCB
understands ``mm'' and ``cm'' as metric units. It also understands
``in,'' ``inch'', and ``"'' to mean inches. If you leave out a unit,
the units of the current grid determine how your input is interpreted.

At the very bottom of the Properties bar, there are two more rows of
buttons: The first allows rotating and flipping of selected objects in
Select mode. It doubles to determine the orientation of newly created
objects in Text and Arc modes. The second row determines on what layer
newly created objects appear and can also be used to move objects
between layers.

\subsection{The Status bar}
The status bar displays the coordinates of the cursor and shows the
identity of holes, pads, and components that you hover over. It is
also home to a popup menu for selecting grid spacing. (Several common
choices prepopulate the menu, but you can also define your own custom
grid spacing in either inches or millimeters.) Lastly, a group of
icons on the right of the Status bar serves as indicators and toggles
for visibility of (from left to right): The silkscreen layer; the top
copper layer; the bottom copper layer; filled planes; and nets.
The
last two deserve further explanation:
\begin{itemize}
  \item Filled planes may be placed on
either top or bottom copper layers. They are only visible when both
the layer on which they occur is visible, and the ``filled planes
visibility'' toggle is on.
\item ``Nets'' are collections of traces, holes, and pads that are
  electrically connected. When nets are visible, these collections are
  highlighted when the mouse hovers over any of their
  members. Additionally, if your PCB design is linked to a CSchem
  schematic, holes or pads that should be part of a net (but
  aren't yet) are highlighted in blue, and holes or pads that should
  not be part of a net (but are) are highlighted in pink. The message
  area on the left of the status bar may also show pertinent
  information about nets.
\end{itemize}

\subsection{Parts pane}

When a PCB layout is linked to a CSchem schematic, one additional user
interface component is shown alongside the PCB: A list of all the
components of the schematic that have not yet been placed on the
PCB. The function of this pane is explained in the next section.

\section{Working with components}
CPCB ships with a small library of predefined components. Press
``Control''+``Shift''+``O'' to show the library (or select ``Open
library'' from the Tools menu). CPCB components are saved as SVG
files, so in some operating systems the Filer window will show
previews. Components may be placed on the PCB simply by dragging them
in from a Filer window. Please note that CPCB cannot handle arbitrary
SVG files, only the special SVG files that it creates itself.

Components may also be dragged onto the ``Parts'' pane
\end{document}
